\begin{abstract}
\paragraph*{Abstract (AW)}
$\;$ \\
$\;$ \\
% Ziel der Arbeit
Dieses Gutachten hat zum Ziel, das Informationsmanagement an der Hochschule 
Emden/Leer zu untersuchen und eine potenzielle Neuordnung vorzuschlagen. 
Dies erfolgt in dem Bestreben, die Informationsinfrastruktur an ein geändertes 
Nutzungsverhalten des Personals und der Studierenden anzupassen und ein 
zeitlich und finanziell effizienteres Informationsmanagement zu betreiben.
	
% Wiedergabe der Struktur
% Grundlagen 1.1
Im Grundlagenkapitel werden die Grundbegriffe des Informationsmanagements 
erläutert und die herrschenden Meinungen zu diesem Thema nach 
\textit{Heinrich}, \textit{Wollnik} und \textit{Krcmar} vorgestellt, wobei das Werk Krcmars umfangreicher dargestellt wird. Als Ziele des Informationsmanagements werden 	zum einen die Koordination der Informationslogistik und zum anderen die Unterstützung der Unternehmensziele durch die Informatik genannt.
	
% Grundlagen 1.2
Im Anschluss an das Grundlagenkapitel werden gegenwärtige Trends des
Informationsmanagements betrachtet, wobei zunächst auf organisatorische
Trends eingegangen wird.
	
Dargestellt werden hier zum einem die Serviceorientierung, die nach den 
Prinzipien der IT Infrastructure Library (ITIL) und unter Einsetzung eines Chief 
Information Officers (CIO) erfolgen soll und die Ausrichtung von 
Dienstleistungen auf die Anforderungen der Kunden zum Gegenstand hat. 
	
Zum anderen wird die Prozessorientierung dargestellt, die eine Ausrichtung der
Organisation nach durchgehenden Geschäftsprozessen zum Ziel hat.
Das Hochschulrechenzentrum bevorzugt eine prozessorientierte Ausrichtung.

Das Kapitel umfasst weiter die Betrachtung von Trends unter der Überschrift
Neue Medien. Die Schwerpunkte liegen hier einerseits auf der Außendarstellung der Hochschule über ein responsive Website, andererseits auf (mobilen) Apps als Informationssystem.
	
% Grundlagen 1.3
Das Kapitel Best-Practice-Beispiele von Informationsmanagement an Hochschulen
betrachtet praktische Implementierungen von Informationsmanagement an den Hochschulen WWU Münster, TU Dortmund, Karlsruher Institut für Technologie und Universität Ulm. Gemeinsame Aspekte aller betrachteten Projekte sind die
Zentralisierung von Diensten, Nutzer- und Serviceorientierung und der Einsatz eines Gremiums mit der Funktion eines CIO.

% Ist-Zustand 
Im weiteren Verlauf der Arbeit wird der gegenwärtige Stand des 
Informationsmanagements an der Hochschule Emden/Leer analysiert.
Ansätze für Änderungen werden in einem zentralen System für den direkten Zugriff 
auf angeschlossene Informationssysteme und in der Besetzung eines 
übergeordneten Informationsmanagers (CIO) gesehen.

% Soll-Konzept
Aufbauend auf der Ist-Analyse und den genannten Trends und Best-Practices werden 
Änderungsvorschläge erarbeitet. Das entwickelte Soll-Konzept setzt hierbei 
Schwerpunkte in der Erweiterung von externen und internen Marketingmaßnahmen,
in der Einführung einer zentralen Logdatei für Supportmaßnahmen sowie daraus 
abzuleitenden Feedbackmaßnahmen zur
Prozessverbesserung und in der Optimierung von Hard- und Software mit dem immer wiederkehrenden Aspekt eines Single-Sign-Ons. Weiter wird die Einführung eines 
Gremiums empfohlen, dass die Funktion eines CIO wahrnimmt und organisatorisch zwischen dem Präsidium und den einzelnen Fachbereichen sowie der Pressestelle positioniert ist.
	
% Ergebnisse Gruppe 4.1
Die nachfolgende Umsetzungsplanung stellt zum einen das Change-Management zur Begleitung der vorgeschlagenen Veränderungen vor und nennt zum anderen zwei Migrationsbeispiele für die Einführung neuer Software. 

Die Betrachtung des Change-Managements führt aus, dass erfolgreiche 
Veränderungspro-zesse durch einen hohen Grad an Information und Partizipation der
von den Veränderun-gen betroffenen Personen geprägt sind.

Die Migrationsbeispiele führen einerseits die Einführung der 
Dokumentenmanagementsoftware Alfresco aus, andererseits das Update des verwendeten
Content-Management-Systems TYPO3 auf eine aktuelle Version.
	
% Ergebnisse Gruppe 4.2
Im letzten Kapitel werden mit der TCO-Methode (Total cost of ownership) und dem Gantt-Diagramm sowie der Meilensteintrendanalyse Methoden zur Schätzung finanzieller bzw. zeitlicher Aufwendungen vorgestellt und auf Beispiele in Bezug auf die Hochschule angewendet.

Das erste Beispiel stellt die Einführung der 
Dokumentenmanagementsoftware Alfresco dar, die mit Kosten von gut 200.000 EUR über
die ersten vier Jahre beziffert wird. Der zeitliche Aufwand wird mit 30 MT (Mitarbeiterinnen- und Mitarbeitertage) geschätzt.

Als zweites Beispiel wird das Redesign/Relaunch der Hochschul-Website angeführt.
Die Schätzung führt hier zu einem zeitlichen Aufwand von ca. 50 MT für einen 
professionellen Dienstleister. Die tatsächlichen Kosten hierfür schwanken aufgrund
regionaler Unterschiede.

Ein letztes Beispiel wird mit der Erstellung von sog. Page Tabs für das Facebook-Profil
der Hochschule dargestellt. Unter Wiederverwendung der Ergebnisse des Website-Redesigns führt hier die Aufwandsschätzung zu insgesamt 22 MT.

Den Schluss des Gutachtens bildet eine Vorstellung von Projekten, die erfolgreich ein
umfassendes Informationsmanagement an Hochschulen eingeführt haben und dafür mit
Fördergeldern der Deutschen Forschungsgemeinschaft ausgestattet wurden, sowie eine
Schätzung von möglichen Fördergeldern in Bezug auf die Hochschule Emden/Leer.

\end{abstract}