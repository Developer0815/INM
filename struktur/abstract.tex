\begin{abstract}
\paragraph*{Abstract (AW)}
$\;$ \\
$\;$ \\
% Ziel der Arbeit
Dieses Gutachten hat zum Ziel, das Informationsmanagement an der Hochschule 
Emden/Leer zu untersuchen und eine potenzielle Neuordnung vorzuschlagen. 
Dies erfolgt in dem Bestreben, die Informationsinfrastruktur an ein geändertes 
Nutzungsverhalten des Personals und der Studierenden anzupassen und ein 
zeitlich und finanziell effizienteres Informationsmanagement zu betreiben.
	
% Wiedergabe der Struktur
% Grundlagen 1.1
Im Grundlagenkapitel werden die Grundbegriffe des Informationsmanagements 
erläutert und die herrschenden Meinungen zu diesem Thema nach 
\textit{Heinrich}, \textit{Wollnik} und \textit{Krcmar} vorgestellt, wobei das Werk Krcmars umfangreicher dargestellt wird. Als Ziele des Informationsmanagements werden 	zum einen die Koordination der Informationslogistik und zum anderen die Unterstützung der Unternehmensziele durch die Informatik genannt.
	
% Grundlagen 1.2
Im Anschluss an das Grundlagenkapitel werden gegenwärtige Trends des
Informationsmanagements betrachtet. Dabei werden zunächst organisatorische
Trends betrachtet.
	
Dargestellt werden hier zum einem die Serviceorientierung, die nach den 
Prinzipien der IT Infrastructure Library (ITIL) und unter Einsetzung eines Chief 
Information Officers (CIO) erfolgen soll und die Ausrichtung von 
Dienstleistungen auf die Anforderungen der Kunden zum Gegenstand hat. 
	
Zum anderen wird die Prozessorientierung dargestellt, die eine Ausrichtung der
Organisation nach durchgehenden Geschäftsprozessen zum Ziel hat.
Das Hochschulrechenzentrum bevorzugt eine prozessorientierte Ausrichtung.

Das Kapitel umfasst weiter die Betrachtung von Trends unter der Überschrift
Neue Medien. Die Schwerpunkte liegen hier einerseits auf der Außendarstellung der Hochschule über ein responsive Website, andererseits auf (mobilen) Apps als Informationssystem.
	
% Grundlagen 1.3
Das Kapitel Best-Practice-Beispiele von Informationsmanagement an Hochschulen
betrachtet praktische Implementierungen von Informationsmanagement an den Hochschulen WWU Münster, TU Dortmund, Karlsruher Institut für Technologie und Universität Ulm. Gemeinsame Aspekte aller betrachteten Projekte sind die
Zentralisierung von Diensten, Nutzer- und Serviceorientierung und der Einsatz eines Gremiums mit der Funktion eines CIO.

% Ist-Zustand 
Im weiteren Verlauf der Arbeit wird der gegenwärtige Stand des 
Informationsmanagements an der Hochschule Emden/Leer analysiert.
Ansätze für Änderungen werden in einem zentralen System für den direkten Zugriff 
auf angeschlossene Informationssysteme und in der Besetzung eines 
übergeordneten Informationsmanagers (CIO) gesehen.

% Soll-Konzept
 und
 daraus ein Konzept zur Neuordnung entwickelt.
 
 

	
% Ergebnisse Gruppe 3
Das entwickelte Soll-Konzept setzt Schwerpunkte in der Erweiterung von externen und internen Marketingmaßnahmen,
in der Einführung einer zentralen Logdatei für Supportmaßnahmen sowie daraus abzuleitenden Feedbackmaßnahmen zur
Prozessverbesserung und in der Optimierung von Hard- und Software mit dem immer wiederkehrenden Aspekt eines Single-Sign-Ons.
	
% Ergebnisse Gruppe 4.1
Die nachfolgende Umsetzungsplanung konzentriert sich vornehmlich auf das Change-Management zur Begleitung der vorgeschlagenen 
Veränderungen und führt zwei Migrationsbeispiele aus. Das erste Beispiel beschreibt die Einführung einer responsiven Webseite mit 
zugrundeliegendem Content-Management-System TYPO3. Das zweite Beispiel führt die Einführung des Dokumentenmanagementsystems Alfresco aus.
	
% Ergebnisse Gruppe 4.2
Die zum Schluss durchgeführte Kosten- und Zeitschätzung kommt zu dem Ergebnis, dass die Einführung des Dokumentenmanagementsystems
zu Kosten von ca. EUR 210.000 über vier Jahre führen wird.
Das Redesign und der Relaunch der Homepage der Hochschule werden mit mindestens 50 Mitarbeiterinnen- und Mitarbeitertagen beziffert,
die Erstellung einer Facebook-Seite mit 21,5 Mitarbeiterinnen- und 
Mitarbeitertagen.
	
	
	
	
	
\end{abstract}