\chapter{Zusammenfassung - AW}
Dieses Gutachten hatte das Informationsmanagement zum Thema. 
Es wurde am Beispiel der Hochschule Emden/Leer analysiert, welche 
Besonderheiten das Thema in Bezug auf Hochschulen birgt.

Die Hochschule Emden/Leer weist dabei im Vergleich zu anderen 
betrachteten Hochschulen bereits  interessante Ansätze auf, 
hält aber auch noch einiges Verbesserungspotenzial bereit. 

Die erarbeiteten Vorschläge für die Neuordnung des vorherrschenden 
Informationsmanagements an der Hochschule zeigen insbesondere 
in die Richtung einer zentralen Verwaltung und Bereitstellung von 
Informationen sowie eines zentral geregelten als auch mobilen 
Zugriffs auf diese Informationen.

Die zeitlichen und finanziellen Schätzungen zur Einführung der 
Änderungen deuten nicht zu vernachlässigende Aufwendungen an, 
sodass die Realisation der Vorschläge die gründliche Abwägung 
des Nutzens gegenüber den Kosten einbeziehen sollte. 

Dabei wird der Fortschritt hin zu einer modernen und effizienten 
Hochschule immer in  Spannungsfeldern hinsichtlich der Autonomie 
der Fachbereiche und der Wertschätzung des Bildungssystems 
durch die Politik und die damit verbundene finanzielle Ausstattung erfolgen.