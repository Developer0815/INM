% !TEX root = ../../../main.tex
% !TEX encoding = UTF-8 Unicode
% !TEX encoding = UTF-8

\subsection{Kostenarten in der IT}
In einer Hochschule ist ein Rechenzentrum für die Aufgaben der IT zuständig.
Der Rechenzentrumsleiter ist in einem Netzwerk, aus leitenden Personen der Hochschule, das Zentrum der personellen IT-Komponenten. In der IT einer Hochschule werden direkte Kosten zwischen den Primärkategorien Hard- und Software, operativer Betrieb und Verwaltung differenziert.\footcite[494]{hansen_business_2009}

Zusätzlich ist ein Rechenzentrum der Hochschule, nach Interview-Aussage des Rechenzentrumsleiters der Hochschule Emden/Leer, jährlich auf ein bestimmtes Budget festgelegt. Danach ist die Tabelle \ref{tab_auswahl_IT_kostenarten}\footcite[493-498]{hansen_business_2009} zu beachten, welche Kosten budgetiert sind und welche nicht. Die folgenden Tabellen sind für den Rechenzentrumsleiter als Kontrollhilfe angedacht, damit keine IT-Kostenarten unbeachtet bleiben.

\begin{table}[h!]
	\begin{tabularx}{\textwidth}{|X|X|}
		% Überschriften
		\hline \textbf{Budgetierte Kosten}  &  \textbf{Nicht budgetierte Kosten}\\
		% Zeile 1
		\hline Software-Entwicklung 
			\begin{itemize}
				\item Neuentwicklung und Anpassungen
				\item Personal- und Sachkosten	
			\end{itemize}  
		& Negative Produktivitätseffekte
			\begin{itemize}
				\item Antwort-, Rüst- und Bearbeitungszeit
				\item Motivation
				\item Ergonomie
			\end{itemize} \\ 
		% Zeile 2	
		\hline Kommunikation \begin{itemize}
			\item Netzwerk
			\item Personal- und Sachkosten			
		\end{itemize} & Ausfall \begin{itemize}
			\item geplant
			\item ungeplant
		\end{itemize} \\ 
		% Zeile 3
		\hline Hardware / Software \begin{itemize}
			\item Abschreibung, Miete und Leasing
			\item Entsorgung
			\item Client / Server
			\item Administration	
		\end{itemize} & Endbenutzer \begin{itemize}
			\item Peer-Support (selbst/gegenseitig)
			\item Unproduktives Konfigurieren
			\item Qualifizierung (selbst /gegenseitig)
		\end{itemize}  \\
		% Zeile 4
		\hline Support &   \\
		% Zeile 5 
		\hline Systembetrieb und Systemmanagement & \\
		\hline
	\end{tabularx}
	\caption{Auszug der IT-Kostenarten nach Krcmar}
	\label{tab_auswahl_IT_kostenarten}
\end{table}

\clearpage 

Als spezielle IT-Kostenarten werden von Gadatsch und Mayer aufgelistet\footcite[349]{gadatsch_masterkurs_2014}:

\begin{table}[h!]
	\begin{tabularx}{\textwidth}{|l|X|}
		% Überschriften
		\hline \textbf{Sekundäre Kostenarten}  &  \textbf{Primäre Kostenarten}\\
		% Zeile 1
		\hline Hardware-Kosten &
		\begin{itemize}
			\item Miete / Leasing
			\item Hardware
			\item Leitungsgebühren
			\item Wartung
		\end{itemize} \\ 
		% Zeile 2	
		\hline Software-Kosten  & 
		\begin{itemize}
			\item Miete / leasing
			\item Software
			\item eigene Entwicklung
			\item Externe Wartung
			\item Beratung
		\end{itemize} \\ 
	% Zeile 3
	\hline Daten-Kosten &
	\begin{itemize}
		\item Beratung
		\item Kauf
	\end{itemize}  \\
	% Zeile 4
	\hline Sonstige IT-Kosten &
		\begin{itemize}
			\item IT-Verbrauchsmaterial
			\item IT-Versicherungen
			\item Beiträge zu Fachverbänden
			\item IT-Fachliteratur
			\item IT-Schulungen
		\end{itemize}\\
	% Zeile 5 
	\hline Innerbetriebliche IT-Leistungsverrechnung & 
	\begin{itemize}
		\item Umlagen
		\item Entwicklungskosten
		\item Benutzerservice
	\end{itemize}\\
	\hline
\end{tabularx}
\caption{Auflistung der speziellen IT-Kosten, nach Gadatsch \& Mayer}
\label{tab_auflistung_spezielle_IT_Kosten}
\end{table}

Die Kostenarten aus den Auflistungen der Tabellen \ref{tab_auswahl_IT_kostenarten} und \ref{tab_auflistung_spezielle_IT_Kosten} eignen sich laut Hansen\footcite[493-498]{hansen_business_2009} für die Betrachtung der Kosten an einer Hochschule.

Als mögliche Nutzung der Kostenarten, schlägt Krcmar die TCO-Methode als Bewertungstechnik vor, was ausführlicher im Kapitel \ref{kosten_zeit_anwendung} beschrieben und zur Teilkalkulation verwendet wird.\footcite[144]{krcmar_einfuhrung_2015} Die TCO-Methode nutzt die Kostenarten, um die wirtschaftlichen Auswirkungen in der IT aufzuzeigen.

\clearpage

Vor allem im Bezug auf Kostenarten und IT wird als Trend ein IT-Controller empfohlen\footcite[49]{gadatsch_masterkurs_2014}, um eine erfolgreiche Wertschöpfung in der IT zu erreichen. Der IT-Controller hat die Transparenzverantwortung gegenüber dem CIO und entlastet ihn damit, wodurch der CIO sich gezielt auf seine Entscheidungsverantwortung konzentrieren kann.\footcite[Vgl.][]{oncampus_inm_skript_2015} Sollte die Hochschule Emden/Leer dieser Empfehlung folgen, ist besonders auf die klare Rollenverteilung zwischen CIO und IT-Controller zu achten. Ihre Kompetenzen dürfen sich nur in Ausnahmen gegenseitig blockieren. Der CIO sollte im Zweifel eine Entscheidungsgewalt haben und passend dazu die Konsequenzen alleine tragen, wenn er die Entscheidungsgewalt nutzt.\footcite[Vgl.][]{oncampus_inm_skript_2015_2} Im idealen Fall tragen Beide die endgültige Entscheidung in einem Kompromiss.

Gerade in einem solch zentralen Projekt, mit hohem Anteil an IT-lastigen Themen, ist es zumindest empfehlenswert über einen IT-Controller nachzudenken.\footcite[11-15]{stratmann_it_2013} Besonders ist hier auch die höhere Komplexität zu beachten, die jeweils in der IT, DFG geförderter Referenzprojekte, entstand. Einige Referenzprojekte werden später im Kapitel \ref{section_projekt_beispiele} vorgestellt und mit der Hochschule Emden/Leer verglichen. Doch zuvor sollen im Kapitel \ref{section_verfahren_schaetzung} ausgewählte Kosten- und Zeitschätzungen erläutert werden, wie die empfohlenen Kostenarten in der Kalkulation Verwendung finden.

