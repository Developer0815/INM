% !TEX root = ../../../main.tex
% !TEX encoding = UTF-8 Unicode
% !TEX encoding = UTF-8

\subsection{Projektmanagement an einer Hochschule}
\label{subsection_projektmanagement_hochschule}
Wie bereits in Kapitel \ref{chapter_grundlagen_INM} beschrieben, weist eine Hochschule als Organisation eine Reihe von Besonderheiten auf. Für das Projektmanagement bedeutet besonders die Tatsache, dass die einzelnen Fachbereiche ein hohes Maß an Autonomie und Entscheidungskompetenzen besitzen, eine entsprechend angepasste Herangehensweise.\footnote{\cite{hansen_business_2009}}

Die zentrale Herausforderung des Projektmanagements ist es, die Interessen der unterschiedlichen Verwaltungsbereiche, der späteren Nutzer und der Hochschulleitung zu wahren und zu vereinen. Durch die Autonomie der Fachbereiche und deren unterschiedlichen Interessen ist es möglich, dass sich innerhalb der Hochschule konkurrierende Arbeitsgruppen bilden. Es ist daher eine weitere, nicht zu unterschätzende, Aufgabe des Projektmanagements, die Kommunikation zwischen allen beteiligten Arbeitsgruppen, möglichen externen Akteuren und dem akademischen Bereich aufrecht zu erhalten und zu fördern.\footnote{\cite{altvater_organisation_2007}}

Des Weiteren führen umfangreiche Änderungen in Organisationen oftmals zu einer besonderen Eigendynamik, die, im Zusammenspiel mit den aufgeführten Besonderheiten einer Hochschule, zu nicht kalkulierbaren oder unvorhersehbaren Geschehnissen führen können. Der Umstand, dass zu Projektbeginn in der Regel noch nicht alle, für eine exakte Planung benötigten, Informationen zur Verfügung stehen, erschwert zusätzlich die zufriedenstellende Organisation des Projektverlaufs. Um dem entgegen zu wirken ist es empfehlenswert, dass Projekt in iterativ-reflexiven Schleifen mit ausreichender Flexibilität durchzuführen.\footnote{\cite{hansen_business_2009}}

Durch die gewonnene Flexibilität sind Anpassungen während der Ausführung des Projektes möglich und auf besondere Befindlichkeiten kann eingegangen werden. Durch eine iterative Durchführung wird außerdem dem Vorhaben Rechnung getragen, dass nach jeder Umsetzung einer Komponente über den weiteren Verlauf des Projekts reflektiert werden kann.

