\chapter{Einleitung - AW}
\textit{Autor: Andreas Willems}

\glqq Ziel des Informationsmanagements ist es, den bestmöglichen Einsatz der 
Ressource Information zu gewährleisten.\grqq{} So bringt Helmut Krcmar den 
Anlass dieses Gutachtens prägnant auf den Punkt.\footcite[11]{krcmar_einfuhrung_2015}

Aus dieser Formulierung lassen sich zwei zentrale Themen des Informationsmanagements ableiten:

Informationen sollen erstens \textbf{effektiv} eingesetzt werden. Es soll erkannt werden, 
welcher Empfänger welche Information zu welchem Zeitpunkt benötigt und aus welcher 
Quelle diese Information bezogen werden kann. Nach Krcmar fällt dies in den Bereich des
\glqq Managements der Informationswirtschaft\grqq{}.
\footcite[Vgl.][13 ff.]{krcmar_einfuhrung_2015}

Zweitens sollen Informationen \textbf{effizient} eingesetzt werden. Die Bereitstellung vorhandener und die Erlangung neuer Informationen soll mit möglichst geringem Aufwand
erreicht werden. Hierzu bedarf es eines \glqq Managements der Informationssysteme\grqq{}
\footcite[Vgl.][41 ff.]{krcmar_einfuhrung_2015}
und eines \glqq Managements der Informations- und Kommunikationstechnik\grqq{}
\footcite[Vgl.][89 ff.]{krcmar_einfuhrung_2015}

Im Rahmen dieses Gutachtens soll untersucht werden, wie sich das Informationsmanagement
an der Hochschule Emden/Leer neu ordnen und verbessern ließe.

Den Anlass hierzu bieten zum einen geänderte Gewohnheiten in der Beschaffung und 
Nutzung von Informationen. So nutzten 2014 bereits 92 Prozent der 14-29 jährigen Onlineanwendungen
für die Suche nach Informationen.\footcite{ardzdf_studie_2014}
Weiter ist, bedingt durch die hohe Verbreitung von internetfähigen Smartphones, die Nachfrage nach speziell 
aufbereiteten, an die Größe und Bedienbarkeit der Geräte angepasste Inhalte stark gestiegen. Nach Zahlen des 
Statistischen Bundesamtes verfügten in 2014 93,6 Prozent der deutschen Haushalte über 
Mobiltelefone.\footcite{statistisches_bundesamt_2015}

Zum anderen bedingt die Anpassung und Optimierung des Informationsmanagements eine schnellere und 
effizientere Verarbeitung von Informationen.

Dieses Gutachten ist in insgesamt neun Kapitel unterteilt und enthält neben dieser Einleitung als erstem Kapitel und einer Zusammenfassung im neunten Kapitel folgende Abschnitte:

In Kapitel 2 werden allgemeine Begriffe des Informationsmanagements erläutert und
besondere Aspekte des Informationsmanagements an Hochschulen betrachtet.

Das Kapitel 3 gibt eine Übersicht über aktuelle Trends des Informationsmanagements an Hochschulen, deren 
Integration in das Informationsmanagement teilweise an Beispielen an den Hochschulen in Münster, Dortmund, 
Karlsruhe und Ulm im Kapitel 4 dargestellt wird.

In Kapitel 5 wird der Ist-Zustand hinsichtlich des Informationsmanagements an der Hochschule Emden/Leer erfasst. Hierzu werden unter Einbeziehung des Leiters des Rechenzentrums der Hochschule Emden/Leer bestehende Informationssysteme betrachtet.

Das Kapitel 6 nennt Soll-Konzepte für verschiedene Bereiche, in denen eine Anpassung des 
Informationsmanagements als erforderlich angesehen wird.

Die Kapitel 7 und 8 schließlich zeigen mögliche Wege auf, wie der gewünschte Soll-Zustand erreicht werden kann
und nennen dabei auch zeitliche und finanzielle Anforderungen an die vorgeschlagenen technischen und 
organisatorischen Änderungen.

Die Erstellung dieses Gutachtens erfolgt als Seminararbeit im Rahmen des Moduls \textit{Informationsmanagement} an der Hochschule Emden/Leer im Sommersemester 2015. 