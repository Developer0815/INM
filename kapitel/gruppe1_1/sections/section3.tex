\section{Grundlegende Aufgaben des Informationsmanagements - MiB}
\label{grundlegende_aufgaben_des_informationsmanagements}
Bei der Einführung eines Informationsmanagements zählen die vorausschauende 
Planung, welche Hard- und Software-Ressourcen von Nöten sind und die gezielte 
Implementierung dieser im Unternehmen zu den grundlegenden Aufgaben. 
Weiterhin wird die Qualität der zu vermittelnden Informationen und im Folgekapitel 
\ref{section_qualitaetsmanagement_der_informationsprozesse} auch der Kommunikationswege von Informationssender zu -empfänger genauer beleuchtet, 
um Ansatzpunkte zu liefern, in welchen unternehmensspezifischen Prozessen 
Handlungsbedarf bestehen könnte.

\subsection{Modellierung der Informationslogistik}
In Anlehnung an das Kapitel \ref{subsection_koordination_informationslogistik} wird im Folgenden die Koordination der Informationslogistik um praxisnahe Anwendungsbeispiele bei der Einführung von 
Hard- und Softwarekomponenten erweitert. Des Weiteren werden besondere Anforderungen an 
Hochschulen aufgezeigt, die bei der Auswahl von neuen Computersystemen entstehen können, 
um vor eventuell auftretenden Problematiken bereits im Vorfeld zu warnen.

\subsubsection{Hardware}
Um eine stetige Verfügbarkeit und Sicherheit von zentral gelagerten Daten zu gewährleist-en, 
sollte ein Unternehmen gewisse Mindestanforderungen an die eingesetzten Hardwarekomponenten 
und ihre Leistung erfüllen. 

Krcmar unterscheidet im Bereich der Hardware zwei grundlegende Faktoren, die in ausreichender 
Menge und Qualität vorhanden sein müssen: Speicher- und Kommunikationskapazität. 

Unter der Speicherkapazität wird die maximal zur Verfügung stehende Menge an Datenspeicher verstanden. 
Ab einer bestimmten Unternehmensgröße sollte im Unternehmen ein Raum existieren, der genug Platz 
für mehrere Server bietet. Bei der Anschaffung von Datenträgern sollte der zu erwartende 
Datenzuwachs im Unternehmen und auch die Vergrößerung einzelner Dateien im Laufe weniger Jahre 
bedacht werden und eine ausreichend große Menge an Speicherplatz als Puffer eingeplant werden. 

Die Kommunikationskapazität beschreibt die Leistungsfähigkeit eines Hardwaresystems in Bezug auf 
die Nutzbarkeit des Netzwerks von einer Vielzahl an Menschen, die im Extremfall alle zum gleichen 
Zeitpunkt Zugriff auf den Datenbestand haben müssen, ohne die Netzwerkleistung negativ zu 
beeinträchtigen.\footcite{krcmar_einfuhrung_2015}

In vielen Unternehmen sollte zudem vor Anschaffung entsprechender Hardware überlegt werden, ob 
es nötig ist, von außerhalb des Unternehmens auf den Server zugreifen zu können. Dies bewirkt 
zwar in der Regel keine Vermehrung der Nutzer, stellt jedoch technisch andere Herausforderungen 
an die Hardware, deren Lösung in einem solchen Fall bereits im Voraus implementiert werden sollte.

Die Besonderheit im Bereich von Hochschulen stellt eindeutig die maximale Kapazität des Netzwerks 
da. Während des Semesters kommt es immer wieder zu Stoßzeiten, in denen Lehrende, 
Verwaltungsangestellte und auch anwesende Studierende zeitgleich über mindestens ein Endgerät 
auf das Netzwerk zugreifen wollen. Für diese besondere Herausforderung muss das Netzwerk 
ausgerichtet sein und mit Stabilität überzeugen.

Die Sicherheit der Daten ist ein wichtiger Aspekt, der nicht in Vergessenheit geraten darf. 
Beispielsweise ist die Wahl der Positionierung des Serverraums nicht ganz unerheblich, je nördlicher 
der Standort des Unternehmens innerhalb Deutschlands liegt. Überschwemmungen bilden eine große 
Gefahr für im Keller gelagerte Serverräume und könnten binnen weniger Stunden den kompletten 
Datenbestand vernichten.

Zudem sollte ein Backup-System eingeführt werden, welches in regelmäßigen Abständen mehrfach 
Kopien des gesamten Datensatzes erstellt. Dieses Backup sollte optimalerweise nicht an gleichem 
Standort wie der Original-Datensatz aufbewahrt werden.

\subsubsection{Software}
Für die Realisierung eines funktionierenden Informationsmanagements in einem Unternehmen ist es 
nach Anschaffung und Installation benötigter Hardware wichtig, eine gelungene und gezielte Auswahl 
der in Zukunft zu nutzenden Software zu treffen. Im Gegensatz zum im Kapitel 
\ref{section_aufbau_des_informationsmanagements_nach_krcmar} beschriebenen 
Software-Entwicklungsprozess für die eigenständige Programmierung benötigter Software werden im 
Folgenden die Besonderheiten im Prozess der Auswahl und Bewertung von Drittanbieter-Software für 
ein Informationsmanagement erörtert.

Ohne genaue Analyse im Voraus, für die sich das Unternehmen unbedingt etwas Zeit nehmen sollte, 
besteht die Gefahr, dass die neue Software nicht in vollem Umfang den Bedarf an Funktionalitäten 
abdeckt, um im Unternehmen ganzheitlich mit einem ausgereiften Informationsmanagement zu arbeiten. 
Eine Abänderung der gewählten Software im Nachhinein ist nicht nur kosten- und zeitintensiv, sondern 
im schlimmsten Fall gar unmöglich, was den kompletten Analyse-Prozess, welche Software die 
richtige für das Unternehmen ist, von vorn beginnen lässt.

Zur Auswahl einer geeigneten Standardsoftware für das Management der Informationen sollten konkret 
mehrere Phasen durchlaufen werden:\footcite{gronau_auswahl_2001}

Zu Beginn des Entscheidungsprozesses werden die Ziele, die mit der Benutzung der Software verfolgt werden, 
definiert. Die Zieldefinition sollte hierbei unter anderem die Ausgangssituation, angestrebte Verbesserungen 
in organisatorischer und technischer Hinsicht, Zieltermin sowie das einzusetzende Budget enthalten.

In einem weiteren Schritt werden die Anforderungen an die einzusetzende Software in Themenbereiche 
gegliedert und nach Prioritäten sortiert. Gerade die Sortierung nach Notwendigkeit an dieser Stelle ist 
von enormer Wichtigkeit, um eine geeignete Software zu finden.

Sind die Anforderungen an die Software zusammengestellt, gilt es nun in einem nächsten Schritt, 
den Markt nach potentiellen Softwareangeboten zu durchsuchen und in Frage kommende Angebote 
zu selektieren. Die Softwarehersteller der engeren Auswahl sollten in jedem Fall mit einer 
Software-Präsentation im Unternehmen vorstellig werden, um Rückfragen beantworten zu können 
und die Eignung der Software für das Unternehmen überprüfbar zu machen.

Zu guter Letzt werden nach Auswahl einer geeigneten Software die Vertragsbedingungen mit dem 
Anbieter in Bezug auf Leistungsbeschreibungen, Vergütung, Organisations- und Abnahmeregelungen 
und auch Service- und Wartungsverträge aufgesetzt. 

\subsubsection{Besondere Anforderungen an Hochschulen}
Die besondere Schwierigkeit an Hochschulen besteht in der fachbereichsspezifischen Steuerung der 
Informationslogistik. Nicht jeder Fachbereich möchte sich eventuell mit der von der Hochschulleitung 
angeschafften Hard- und Software zufrieden geben und anfreunden. Gerade technisch orientierte 
Studiengänge möchten verständlicherweise ihr eigenes Know-how nutzen, um den Fachbereich 
in ihrem Sinne mit einer gut funktionierenden Informationslogistik zu steuern.

Um Konflikte bei solchen \glqq Alleingängen\grqq{} zu vermeiden, sollte in der Planungsphase vor 
Hard- und Softwareanschaffung insbesondere die Meinung der technisch orientierten 
Fachbereiche hinzugezogen werden. 

In einem zum späteren Zeitpunkt ganzheitlich funktionierenden Informationsmanagement 
können die Prozesse nur reibungslos funktionieren, wenn jeder Fachbereich die Ressourcen 
der zentralen Hard- und Software für sich zu nutzen weiß.

\subsection{Management der Schnittstellen zu den Informationsempfängern}
\label{subsection_management_schnittstellen_infoempfangern}
Die Schnittstelle beschreibt in diesem Zusammenhang den „Berührungspunkt“ in dem 
die Informationen ausgetauscht werden. Die beteiligten Individuen sind in diesem Fall 
Personen, die über technische Kommunikationsmittel Informationen erhalten oder senden. 
Beispielsweise bekommen Studenten Informationen von ihrem Tutor, über Tag und Uhrzeit 
des nächsten stattfindenden Tutoriums. 

Es handelt sich hier um eine zweiseitige Mensch-Computer-Interaktion (Smartphone, Tablet, 
Laptop hier synonym verwendet), da die Individuen über eine Benutzerschnittstelle ihres 
Computers jeweils miteinander über technische Hilfsmittel Informationen austauschen.
Damit eine Benutzerschnittstelle für den Menschen nutzbar und sinnvoll ist, muss sie auf 
seine Bedürfnisse und Fähigkeiten angepasst sein. Eine gewisse Grundlagenkenntnis im 
technischen Umgang, sowie mit Social-Media- oder Forennutzung, wird in diesem Fall, 
im Rahmen der Digital-Natives-Generation, vorausgesetzt.\footcite{wikipedia_digitalnative_2015}

Somit ist die Voraussetzung des verständlichen Umgangs mit Informationsmedien erfüllt, 
sodass im nächsten Schritt dafür gesorgt werden muss, dass Informationen vorhanden sind, 
die übermittelt werden können. Beispielhaft wäre es hier anzunehmen, dass ein Informatikstudent, 
durch die Zugehörigkeit in seinem Fachbereich und seinem entsprechenden Studiengang durch 
Hochschulpersonal fachbereichsbezogene Informationen durch den Zugang zum Informationsportal 
der Hochschule erhält. Dies kann durch einen E-Mailverteiler oder eventuell durch ein 
Informationssystem mit entsprechenden Zugangsvoraussetzungen (wie Immatrikulation) gewährleistet 
werden. Das verwendete Informationsmedium stellt hier die Schnittstelle zwischen Mensch und der 
Ressource Information dar und bietet die Möglichkeit für den Studenten, gewünschte Informationen, 
die ihn betreffen, zu erhalten.

Auch hier ist das Qualitätsmanagement der zu verwaltenden Informationen ein essentielles Thema. 
Im Beispiel des Informatikstudenten sind für ihn die Informationen interessant, die ihn betreffen. 
Die Informationen zum Studiengang \glqq Hispanistik\grqq{} sind für ihn irrelevant. 
Somit müssen Informationen, im Rahmen der Schnittstellenbetreuung, für die Empfänger vorselektiert 
werden, um keine Informationsüberflutung zu provozieren oder zu vermeiden, dass die übermittelte 
Information den Informationsbedarf nicht vollständig deckt und dadurch Rückfragen offen bleiben. 
Es ist also darauf zu achten, dass eine Information nicht vorzeitig veröffentlicht wird, ohne dass der 
Inhalt geprüft wurde. Als Beispiel könnte hier die Information \glqq Der Kurs findet heute nicht statt.\grqq{}
betrachtet werden, die für den Studenten zwar eine Teilinformation enthält, den Informationsbedarf 
aber nicht abdeckt, sodass eine Rückfrage entsteht, die zusätzlich verwaltet werden muss. 
Hochschulindividuell kann es auch automatisierte Informationssysteme geben, die von mehreren 
Plattformen zugreifbar sind, wobei auch hinter dieser Automatisierung Personal steht, das die 
entsprechende Information generiert. Auf das Qualitätsmanagement wird im folgenden noch 
detaillierter eingegangen.

Ein weiterer wichtiger Punkt ist, dass die Möglichkeit des Feedbacks seitens des Informationsempfängers 
gewährleistet sein muss. Ob dies nun in Form von Kontaktformularen oder über eine, zur Verfügung 
gestellte, E-Mail-Adresse geschieht, ist hochschulintern individuell. 
Von großer Bedeutung ist dieser Punkt, weil es beispielsweise im Falle einer nicht bedarfsdeckenden 
Information zu Verwirrungen und Missverständnissen kommen kann. 
Die Informationsempfänger brauchen also die Möglichkeit zur Rücksprache, um die Motivation nicht zu 
beeinträchtigen und ggf. Stresssituationen zu umgehen. 
Das Spektrum der Feedbackmöglichkeit ist mit der Möglichkeit der direkten Rücksprache noch nicht 
ausgeschöpft. 
Es gibt viele Möglichkeiten Feedback zu geben bzw. zu erhalten, wie beispielsweise eine Evaluationsdurchführung o.ä..
