% This file was converted to LaTeX by Writer2LaTeX ver. 1.4
% see http://writer2latex.sourceforge.net for more info
\documentclass{article}
\usepackage[utf8]{inputenc}
\usepackage[T1]{fontenc}
\usepackage[ngerman]{babel}
\usepackage{amsmath}
\usepackage{amssymb,amsfonts,textcomp}
\usepackage{array}
\usepackage{hhline}
\title{}
\begin{document}
\section[2.3\ \ Grundlegende Aufgaben des Informationsmanagements]{2.3\ \ Grundlegende Aufgaben des
Informationsmanagements}
Zu den grundlegenden Aufgaben bei Einführung eines Informationsmanagements zählen die vorausschauende Planung, welche
Hard- und Software-Ressourcen von Nöten sind und die gezielte Implementierung dieser im Unternehmen. Weiterhin wird die
Beurteilung der Qualität der zu vermittelnden Informationen und deren Kommunikationswege von Informationssender zu
-empfänger genauer beleuchtet, um thematische Ansatzpunkte zu liefern, an welchen Stellen es Handlungsbedarf in
unternehmensspezifischen Prozessen geben könnte. Eine praxisnahe Anwendungsmöglichkeit der beschriebenen theoretischen
Vorgehensweisen rundet das nachfolgende Kapitel ab. 

\subsection[2.3.1 \ \ Modellierung der Informationslogistik]{2.3.1 \ \ Modellierung der Informationslogistik}
In Anlehnung an das Kapitel 2.1.7 wird im Folgenden die Koordination der Informationslogistik um praxisnahe
Anwendungsbeispiele bei der Einführung von Hard- und Softwarekomponenten erweitert. Des Weiteren werden besondere
Anforderungen an Hochschulen aufgezeigt, die bei der Wahl von neuen Computersystemen entstehen können, um vor eventuell
auftretende Problematiken bereits im Vorfeld zu warnen. \ \ 


\bigskip

2.3.1.1 Hardware


\bigskip

Um eine stetige Verfügbarkeit und Sicherheit von zentral gelagerten Daten zu gewährleisten, sollte ein Unternehmen
gewisse Mindestanforderungen an eingesetzte Hardwarekomponenten und ihre Leistung erfüllen. 


\bigskip

Krcmar unterscheidet im Bereich der Hardware zwei grundlegende Faktoren, die in ausreichender Menge und Qualität
vorhanden sein müssen: Speicher- und Kommunikationskapazität. Unter der Speicherkapazität wird die maximal zur
Verfügung stehende Menge an Datenspeicher verstanden. Ab einer bestimmten Unternehmensgröße sollte in der Firma ein
Raum existieren, der genug Platz für mehrere Server bietet. Bei der Anschaffung von Datenträgern sollte der zu
erwartende Datenzuwachs im Unternehmen und auch die Vergrößerung einzelner Dateien im Laufe weniger Jahre bedacht
werden und eine ausreichend große Menge an Speicherplatz als Puffer eingeplant werden. Die Kommunikationskapazität
beschreibt die Leistungsfähigkeit eines Hardwaresystems in Bezug auf die Nutzbarkeit des Netzwerks von einer Vielzahl
an Menschen, die im Extremfall alle zum gleichen Zeitpunkt Zugriff auf den Datenbestand haben müssen, ohne die
Netzwerkleistung zu beeinträchtigen. \footnote{Krcmar 2015.}


\bigskip

In vielen Unternehmen sollte zudem vor Anschaffung entsprechender Hardware überlegt werden, ob es nötig ist, von
außerhalb des Unternehmens auf den Server zugreifen zu können. Dies bewirkt zwar in der Regel keine Vermehrung der
Nutzer, stellt jedoch technisch andere Herausforderungen an die Hardware, deren Lösung in einem solchen Fall bereits im
Voraus implementiert werden sollte.


\bigskip

Die Besonderheit im Bereich von Hochschulen stellt eindeutig die maximale Kapazität des Netzwerks da. Während des
Semesters kommt es immer wieder zu Stoßzeiten, in denen Lehrende, Verwaltungsangestellte und auch anwesende Studierende
zeitgleich über mindestens ein Endgerät auf das Netzwerk zugreifen wollen. Für diese besondere Herausforderung muss das
Netzwerk ausgerichtet sein und mit Stabilität überzeugen. 


\bigskip

Die Sicherheit der Daten ist ein wichtiger Aspekt, der nicht in Vergessenheit geraten darf. Beispielsweise ist die Wahl
der Positionierung des Serverraums nicht ganz unerheblich, je nördlicher der Standort des Unternehmens innerhalb
Deutschlands liegt. Überschwemmungen bilden eine große Gefahr für im Keller gelagerte Serverräume und könnten binnen
weniger Stunden den kompletten Datenbestand vernichten. 


\bigskip

Zudem sollte ein Backup-System eingeführt werden, welches in regelmäßigen Abständen mehrfach Kopien des gesamten
Datensatzes erstellt. Dieses Backup sollte optimalerweise nicht an gleichem Standort wie der Original-Datensatz
aufbewahrt werden. 


\bigskip

2.3.1.2 Software


\bigskip

Für die Realisierung eines funktionierenden Informationsmanagements in einem Unternehmen ist es nach Anschaffung und
Installation benötigter Hardware wichtig, eine gelungene und gezielte Auswahl der in Zukunft zu nutzenden Software zu
treffen. Im Gegensatz zum im Kapitel 2.2.1 beschriebenen Software-Entwicklungsprozess für die eigenständige
Programmierung benötigter Software werden im Folgenden die Besonderheiten im Prozess der Auswahl und Bewertung von
Drittanbieter-Software für ein Informationsmanagement erörtert. 


\bigskip

Ohne genaue Analyse im Voraus, für die sich das Unternehmen unbedingt etwas Zeit nehmen sollte, besteht die Gefahr, dass
die neue Software nicht in vollem Umfang den Bedarf an Funktionalitäten abdeckt, um im Unternehmen ganzheitlich mit
einem ausgereiften Informationsmanagement zu arbeiten. Eine Abänderung der gewählten Software im Nachhinein ist nicht
nur kosten- und zeitintensiv, sondern im schlimmsten Fall gar unmöglich, was den kompletten Analyse-Prozess, welche
Software die richtige für das Unternehmen ist, von vorn beginnen lässt. 


\bigskip

Zur Auswahl einer geeigneten Standardsoftware für das Management der Informationen sollten konkret mehrere Phasen
durchlaufen werden: \footnote{Gronau 2001.}


\bigskip

Zu Beginn des Entscheidungsprozesses werden die Ziele, die mit der Benutzung der Software verfolgt werden, definiert.
Die Zieldefinition sollte hierbei unter anderem die Ausgangssituation, angestrebte Verbesserungen in organisatorischer
und technischer Hinsicht, Zieltermin sowie das einzusetzende Budget enthalten. 


\bigskip

In einem weiteren Schritt werden die Anforderungen an die einzusetzende Software in Themenbereiche gegliedert und nach
Prioritäten sortiert. Gerade die Sortierung nach Notwendigkeit an dieser Stelle ist von enormer Wichtigkeit, um eine
geeignete Software zu finden.


\bigskip

Sind die Anforderungen an die Software zusammengestellt, gilt es nun in einem nächsten Schritt, den Markt nach
potentiellen Softwareangeboten zu durchsuchen und in Frage kommende Angebote zu selektieren. Die Softwarehersteller der
engeren Auswahl sollten in jedem Fall mit einer Software-Präsentation im Unternehmen vorstellig werden, um Rückfragen
beantworten zu können und die Eignung der Software für das Unternehmen überprüfbar zu machen.


\bigskip

Zu guter Letzt werden nach Auswahl einer geeigneten Software die Vertragsverhandlung mit dem Anbieter in Bezug auf
Leistungsbeschreibungen, Vergütung, Organisations- und Abnahmeregelungen und auch Service- und Wartungsverträge
aufgesetzt. 


\bigskip

2.3.1.3 Besondere Anforderungen an Hochschulen


\bigskip

Die besondere Schwierigkeit an Hochschulen besteht in der fachbereichsspezifischen Steuerung der Informationslogistik.
Nicht jeder Fachbereich möchte sich evtl. mit der von der Hochschulleitung angeschafften Hard- und Software zufrieden
geben und anfreunden. Gerade technisch orientierte Studiengänge möchten verständlicherweise ihr eigenes Know-how
nutzen, um den Fachbereich in ihrem Sinne mit einer gut funktionierenden Informationslogistik zu steuern. 


\bigskip

Um Konflikte bei solchen „Alleingängen“ zu vermeiden, sollten in der Planungsphase vor Hard- und Softwareanschaffung
insbesondere die Meinung der technisch orientierten Fachbereiche hinzugezogen werden. 


\bigskip

In einem zum späteren Zeitpunkt ganzheitlich funktionierenden Informationsmanagement können die Prozesse nur reibungslos
funktionieren, wenn jeder Fachbereich die Ressourcen der zentralen Hard- und Software für sich zu nutzen weiß. 


\bigskip


\bigskip


\bigskip
\end{document}
