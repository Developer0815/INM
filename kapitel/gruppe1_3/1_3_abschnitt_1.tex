\section{Betrachtete Hochschulen}
Diese Betrachtung konzentriert sich auf vier Hochschulen, die im Leitbild für ein Informationsmanagement der Universität Kassel als Best Practice-Hochschulen genannt werden: Die Westfälische Wilhelms-Universität (WWU) in Münster, die Technische Universität Dortmund, das  Karlsruher Institut für Technologie und die Universität Ulm.

\subsection{WWU Münster}
Die WWU in Münster ist mit über 40.000 Studierenden\footnote{\url{http://www.uni-muenster.de/profil/index.shtml}} die größte der hier betrachteten Hochschulen. In Münster wurde bereits „2003 der IKM-Service institutionalisiert“\footnote{\cite[47]{bode_informationsmanagement_2010}} um „den Anforderungen an ein integriertes Informationsmanagement im Überlappungsfeld von Information, Kommunikation und Medien (IKM)“\footnote{\cite{bode_informationsmanagement_2010}} gerecht zu werden. „In diesem Rahmen wurde das Projekt Münster Information System for Re\-search and Organization (MIRO) entwickelt“\footnote{\cite[47]{bode_informationsmanagement_2010}}, das „über 5 Jahre vor allem mit der Bereitstellung von wissenschaftlichem Personal gefördert“\footnote{\cite[7]{vogl_bericht_2013}} wurde und „nach einer Verlängerung auf sechs Jahre am 31.12.2011 zu Ende“\footnote{\cite[1]{vogl_bericht_2013}} ging. Besonders relevant ist das Projekt MIRO deshalb, weil es ein explizites Ziel des Projektes war, „anderen Hochschulstandorten beispielhaft einen Rahmen aufzeigen, den diese auch ohne DFG-Förderung individuell anwenden oder nachnutzen konnten.“\footnote{\cite[1]{vogl_bericht_2013}}
Bei Betrachtung der Erkenntnisse aus Projekt MIRO sollte jedoch immer beachtet werden, dass die Anforderungen einer Universität der Größe der WWU Münster nicht unbedingt ohne weiteres auf kleinere Hochschulen übertragbar sind.

\subsection{TU Dortmund}
Die Technische Universität Dortmund ist mit rund 32.800 Studierenden\footnote{\url{http://www.tu-dortmund.de/uni/Uni/Profil/index.html}} nur unwesentlich kleiner als die WWU Münster. In Dortmund gibt es das IT \& Medien Centrum (ITMC), das sich als „ganzheitlichen Dienstleister für IT-Aufgaben der Technischen Universität Dortmund“\footnote{\url{http://www.itmc.uni-dortmund.de/beritmc/ueber-itmc.html}} versteht. Dieser ist aus dem Hochschulrechenzentrum und dem Medienzentrum mit dem Ziel entstanden, „die IT-Kompetenzen der zentralen Einrichtungen zu stärken.“\footnote{\url{http://www.itmc.uni-dortmund.de/beritmc/ueber-itmc.html}}

\subsection{Karlsruher Institut für Technologie}
Das Karlsruher Institut für Technologie (KIT) mit über 24.000 Studierenden\footnote{\url{http://www.kit.edu/kit/daten.php}} wurde im Jahr 2009\footnote{\url{http://www.kit.edu/kit/daten.php}} durch den Zusammenschluss der Universität Karlsruhe mit dem Forschungszentrum Karlsruhe gegründet\footnote{\url{http://www.kit.edu/kit/geschichte.php}}.

Am KIT verfolgte das Projekt Karlsruher Integriertes InformationsManagement (KIM) das Ziel, durch Schaffung effizienter organisatorischer Koordinierungs-, Kompetenz- und Servicestrukturen die Zusammenarbeit zwischen den verschiedenen Einrichtungen des KIT zu optimieren, Entscheidungswege zu verkürzen und die Konsistenz der Geschäftspro\-zesse zu erhöhen.\footnote{\url{http://kim.cio.kit.edu}}

Im Rahmen dieses Projektes wurde ein Ausschuss für Informationsversorgung und \mbox{-}\nobreak ver\-arbeitung (AIV) eingerichtet, sowie das Medien- und IV-Service-Centrum Karlsruhe (MICK) gegründet, das die Kompetenzen und Ressourcen des Rechenzentrums, der Universitätsbibliothek, der Medieneinrichtungen und der Verwaltung virtuell zusammenführen soll.\footnote{\url{https://kim.cio.kit.edu/downloads/KIM_UniKaTH061.pdf}}

\subsection{Universität Ulm}
Die Universität Ulm ist mit über 10.000 Studierenden\footnote{\url{http://www.uni-ulm.de/universitaet.html}} die kleinste der im „Leitbild für ein Informationsmanagement der Universität Kassel“ genannten Best Practice-Hochschulen. In Ulm werden im Kommunikations- und Informationszentrum (kiz) „die Kompetenzen rund um die Informations- und Kommunikationsversorgung der Universität gebündelt.“\footnote{\url{https://www.uni-ulm.de/einrichtungen/kiz/wir-ueber-uns.html}}, wobei das kiz die Servicebereiche Bibliothek, Informationstechnik und Medien umfasst.\footnote{\url{https://www.uni-ulm.de/einrichtungen/kiz.html}}
