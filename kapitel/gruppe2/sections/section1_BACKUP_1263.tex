\section{Ziel (TK)}
<<<<<<< HEAD
Mit Hilfe einer Analyse der Ist-Situation an der Hochschule Emden/Leer wird festgestellt, 
in wieweit bereits ein Informationsmanagement besteht. Wenn dies nicht der Fall ist, wird 
recherchiert, welche Informationen zentral gesammelt werden und welche Bereiche in das 
Projekt \textbf{„Potentielle Neuordnung des Informationsmanagements einer kleineren 
Fachhochschule auf der Grundlage bestehender Lösungen an deutschen Hochschulen“} 
mit einbezogen werden müssen.

Wesentliche Fragestellungen, die in diesem Kapitel gelöst werden sollen, sind auf der 
einen Seite, herauszufinden, welche vorhandenen IT-Systeme bereits zentral Verwendung 
finden und auf der anderen Seite, wie Informationen aktuell repräsentiert werden. 

Des weiteren soll in dieser Analyse Aufschluss darüber gegeben werden, ob ein 
Informationsmanagement an der Hochschule betrieben wird und wie Informationen 
bereits zentral zur Verfügung gestellt werden.

Die Hochschule Emden/Leer ist eine kleine Hochschule mit aktuell 4626 eingeschriebenen 
Studierenden. Den größten Anteil machen die 4303 Studenten vor Ort aus.\footnote{\url{http://www.hs-emden-leer.de/fileadmin/user_upload/Einrichtungen/ZDF/Studierende/JV_Stud_20142.pdf}[12.04.2015]} 
Die Hochschule beschäftigt 396 Mitarbeiter.\footnote{\url{https://www.hs-emden-leer.de/no_cache/hochschule/zahlen-daten-fakten.html}[12.04.2015].}
=======
Mit Hilfe einer Analyse der Ist-Situation an der Hochschule Emden/Leer wird festgestellt, in wieweit bereits ein Informationsmanagement besteht. Wenn dies nicht der Fall ist, wird recherchiert, welche Informationen zentral gesammelt werden und welche Bereiche in das Projekt \textbf{„Potentielle Neuordnung des Informationsmanagements einer kleineren Fachhochschule auf der Grundlage bestehender Lösungen an deutschen Hochschulen“} mit einbezogen werden müssen.

Wesentliche Fragestellungen, die in diesem Kapitel gelöst werden sollen, sind auf der einen Seite, herauszufinden, welche vorhandenen IT-Systeme bereits zentral Verwendung finden und auf der anderen Seite, wie Informationen aktuell repräsentiert werden. Des weiteren soll in dieser Analyse Aufschluss darüber gegeben werden, ob ein Informationsmanagement an der Hochschule betrieben wird und wie Informationen bereits zentral zur Verfügung gestellt werden.

Die Hochschule Emden/Leer ist eine kleine Hochschule mit aktuell 4626 eingeschriebenen Studierenden. Den größten Anteil machen die 4303 Studenten vor Ort aus.\footnote{\url{http://www.hs-emden-leer.de/fileadmin/user_upload/Einrichtungen/ZDF/Studierende/JV_Stud_20142.pdf}[12.04.2015]} Die Hochschule beschäftigt 396 Mitarbeiter.\footnote{\url{https://www.hs-emden-leer.de/no_cache/hochschule/zahlen-daten-fakten.html}[12.04.2015].}
>>>>>>> e8844533d562f8340e9cff3ff03f614932467398
