\section{Bewertung und Gewichtung (TK)}
Abschließend kann gesagt werden, dass Informationen zentral gesammelt werden und 
wichtige Systeme wie das "'Laufwerk Y"' und die E-Learning Plattform (Moodle) in allen 
Fachbereichen und in Teilen der Verwaltung zum Einsatz kommen. Durch den starken 
Kooperationsverbund werden zentrale Dienste, wie Springer Link, WISO und video2brain für 
die Studierenden und Mitarbeiter dezentral zur Verfügung gestellt. 

Bei der Repräsentation von Informationen nach außen verfügt die Hochschule über eine 
Pressestelle und eine Marketingabteilung. Es existiert eine feste CD-Reglung für alle 
Abteilungen und Bereiche. 

Durch diverse Arbeitsgruppen ist der Erfahrungs-, Wissens- und Informationsaustausch für 
wichtige Bereiche bereits gegeben. Durch die Arbeitsgruppe ZDF, WEB und Moodle werden 
zentrale Systeme zur Wissenserhaltung und Informationsbereitstellung gepflegt. Dadurch, 
das die Arbeitsgruppen abteilungsübergreifend agieren, besteht auch zwischen den 
einzelnen Bereichen eine Schnittstelle, ohne die autarken Fachbereiche einzuschränken. 

Im Bezug auf Serviceorientierung und IT-Sicherheit lässt sich sagen, dass SSO 
(Single-Sign-On) für einige Bereiche bereits zum Einsatz kommt (siehe Kapitel \ref{realisierung_der_serviceorientierung}). 
Ebenso werden Teile des IT-Grundschutzes erfolgreich an der Hochschule eingesetzt. Dies 
sind erste Schritte zum Informationsmanagement, jedoch fehlt grundsätzlich ein zentrales 
System für den direkten Zugriff und zur Weiterleitung auf weitere Informationssysteme. In 
Kapitel \ref{immatrikulations_und_pruefungsamt} wird beschrieben, dass in Hochschulen, 
welche ein Informationsmanagement einsetzen, dieses meistens im Bereich 
Immatrikulations- und Prüfungsamt (HIS) angesiedelt ist. 

Neben einem  zentralem System fehlt auf der organisatorischen Seite eine Instanz. Wie in 
Kapitel \ref{cio_text} beschrieben, findet im klassischen Informationsmanagement für 
Unternehmen das Management häufig durch einen CIO (Chief Information Officer) statt. In 
Hochschulen wird dies oft durch DACH Organisationen realisiert. 

Auch wenn die Hochschule bereits diverse Arbeitsgruppen einsetzt, so ist diese Instanz des 
Informationsmanagements bisher unbesetzt. Ein Informationsmanagement, wie es in Kapitel 
\ref{begriffsdefintion_inm} beschrieben ist, wird derzeit an der Hochschule nicht vollständig 
praktiziert.